\documentclass{nle}

\usepackage[UTF8]{ctex}
\usepackage{cite}

\title[东北地区黑土流失的现状、原因、解决措施]
      {东北地区黑土流失的现状、原因、解决措施}
\author[KnowsCount]
      {陈泽尘\\
      knowscount@gmail.com\\
      github.com/knowscount}

\pagerange{\pageref{firstpage}--\pageref{lastpage}}
\pubyear{1998}

\bibliography{reference.bib}
\bibliographystyle{acm.bst}

\newcommand\eg{{\it e.g.\ }}
\newcommand\etc{{\it etc}}

\begin{document}

\label{firstpage}
\maketitle

\begin{abstract}
  作为我国重要的粮食产区和商品粮基地,东北黑土区正在经历的水土流失,严重破坏了宝贵的黑土资源,恶化了生态环境,限制了东北社会经济的可持续发展,使其逐渐丧失作为商 品粮基地的“黑土”基础。解决这一问题,对稳定粮食市场、帮张国家粮食安全具有重要作用。~\cite{liubaoyuan:1}。
\end{abstract}

\section{Introduction}

黑土地是一种肥力高、非常适合植物生长的土壤,被誉为粮食生产与供给的“稳压器”。

\subsection{范围与定义}

北半球的三大片黑土,主要位于 40~54°N 间,其中,北美 290 万平方公里,东欧 182 万平方公里,中国东北 35 万平方公里;南半球的一大片黑土位于 27~40°S 间,主要分布于阿根廷潘帕斯大草原,面积 106 万 平方公里。

黑土分布区气候温暖潮湿,夏季雨热 同季,地势起伏较小,适于连片种植。

松辽水利委员会在关于东北黑土区水土流失情况的报告中将黑土区的土壤类型定义为包括黑土、黑钙土 、暗棕壤 、草甸土、白浆土、棕壤、棕色针叶林土、风沙土和沼泽土等。 

\subsection{开发情况}

根据张树文, et al. 的研究结果,东北地区对于黑土地的开发以及利用主要聚集在 20 世纪一来的一百多年。由于清代的满洲统治者视东北为发祥之地的缘故,康熙对其实行了长达两百多年的封禁——使其大部分地区保留成为森林、草原。但在中日战争之后,东北地区便全面解禁,并从此开始大规模地开发满洲土地:东北的耕地面积,从 1683 年的 5396 平方公里(同期垦殖指数 0.7\%),涨到了现在的 21.5 万平方公里~\cite{xinhua:1},几乎是当时的四十倍。迅速的耕地面积提升对土地带来了一些副作用。目前已经能够看到大量切沟——土壤水蚀最严重的阶段——分布,并且大抵四分之一的土地都成了“破皮黄”。

\section{水土流失的现状}

\subsection{沟蚀现状}

作为土地退化最严重的表现形式,沟蚀侵蚀在东北黑土区是中国除黄土高原外沟道侵蚀最为严重的区域 ~\cite{zhangxingyi:1}——60 年的侵蚀沟变化动态结果显示,无论是侵蚀沟条数,还是沟道面积和沟壑密度,都在显著增加~\cite{lizhiguang:1}。

王文娟, et al. 发现,其研究区“结构性侵蚀沟条数为 2246 条,最短侵蚀沟为 19.7 米,最长侵蚀沟为 12499.43 米,侵蚀沟密度为 479.15 米/平方公里,吞噬耕地面积为 1734.05 公顷,破坏耕地面积为 8067.5 公顷;研究区为典型的基质-斑块-廊道农业景观……以上的数字表明研究区正遭受着严重的沟 蚀,严重危害当地粮食生产,必须加大力度治理。”

并且,王文娟等的研究还指出,“研究区为典型的农业景观,旱地占该区面积的 75.72\%,为基质景观类型,其余的景观类型较少,这种景观格局使得该区具有较低的景观分维数和香农多样性 以及较高的聚集度,但以上各指数子流域间差异较小, 这种格局加剧了该区的土壤侵蚀。也就是说集中分布的耕地将是未来水土保持的主战场。”

不过,东北黑土区侵蚀沟总体发育形成近几十年,约 90\% 仍处于发展阶段。

\section{水土流失的原因}

\subsection{退化方式}
中国科学院东北地理与农业生态研究所研究人员观测试验发现,东北黑土带的退化有两种方式,一种是黑土从坡上流到坡下,土壤移动了,造成坡耕地质量下降;另一种方式是耕层土壤的有机质含量下降。

\subsection{人工因素}
旱地较高的耕地率是侵蚀产生的重要原因。

施用有机肥和土壤深松深翻都是保护黑土地的重要举措,但农民对此没有热情——成本太高。一位基层农业干部~\cite{official:1}告诉记者,施用有机肥需要大量的抛撒机械,进口机械太贵,国产的便宜一些,但抛撒效果、效率都达不到进口同类产品的作业标准。

桦川县副县长武庆祥说,国家在土壤深松深翻上没少投入,深耕补贴费由每亩补贴 5 元上升到 10 元。但深松深翻的成本每亩得 50 元至 70 元,仍需要农民拿出很多钱配套。这是深松深翻面积继续扩大的制约因素。

\section{可能的解决措施}

黑土区是国家的重要产量基地,今后一段时间内仍面临着农业生产高压的态势。因而沟道治理不能同黄土高原一样采取退耕还林还草以恢复生态为主的方式,应探索出利用与保护协调可持续发展的方式。

\subsection{植物治沟模式}

植物治沟模式是依照因地制宜,本着“绿色治沟”优先的原则,采用植物措施的侵蚀沟柔性治理的技术体系。原理为利用植物自身的固土功能稳定沟道, 利用植物谷坊对上游来水消能拦沙,栽植植被生态修复。

该模式主要应用于水土条件较好的中小型,以植被措施为主,单条沟治理成本为 7~15 万元。

\subsection{填埋复垦模式}

刘立春在专利 CN103650694B 中提出“一种侵蚀沟复垦方法,它涉及一种水土流失导致农田中形成侵蚀沟的修复技术。它要解决现有耕地侵蚀沟毁损农田,制约大机械的作业,降低了农机耕作效率的问题。方法:一、侵蚀沟沟底修整;二、暗管铺设;三、打桩;四、秸秆打捆;五、秸秆铺设;六、表土填埋;七、间隔筑埂。本发明充分利用秸秆资源,就地取材填埋侵蚀沟,将当前大部分被焚烧的秸秆还田,肥沃土壤,减少二氧化碳的排放,净化了环境,实现侵蚀沟复垦的同时,做到了废弃资源的利用。本发明既复垦了土地,又解决了大型农机行走的问题,提高了农机耕作效率,提高农田水土资源的利用效率,对东北黑土区综合治理,国家粮食生产能力建设,均具有重要的意义。”

这种方法较为具有上文提到的可持续性。

\section{结论}

应该采取适宜且有可持续性的方式,重点保护东北地区珍贵的黑土资源。

\end{document}
